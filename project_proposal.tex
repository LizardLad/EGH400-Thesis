\documentclass[11pt]{article}

\usepackage[a4paper, portrait, margin=2.3cm]{geometry}

\usepackage{fancyhdr}
\usepackage{hyperref}
\setlength{\headheight}{15pt}

\usepackage[table]{xcolor}
\usepackage{float}
\usepackage{svg}

\setlength{\arrayrulewidth}{0.2mm}

\usepackage{tikz}
\usetikzlibrary{shapes, arrows.meta, positioning}

\usepackage{textcomp}
\usepackage{amsmath}
\usepackage{amsfonts}
\usepackage{amssymb}
\usepackage{mathrsfs}
\usepackage{mathtools}
\usepackage{cases}
\usepackage{siunitx}
\usepackage{gnuplottex}

\usepackage{pgfplots}
\usepackage{pdfpages}
\pgfplotsset{width=10cm,compat=1.18}

\usepackage{multicol}

\usepackage{booktabs}
\usepackage{caption}
\usepackage{subcaption}

\usepackage{minted}
\definecolor{light_grey}{gray}{0.9}
\setminted{frame=lines, framesep=2mm, bgcolor=light_grey, fontsize=\footnotesize, linenos}

\usepackage{algorithm}
\usepackage{algpseudocode}

\usepackage{fontspec}
\setmainfont[Path=/usr/share/fonts/TTF/,
    BoldItalicFont=calibri-bold-italic.ttf,
    BoldFont      =calibri-bold.ttf,
    ItalicFont    =calibri-italic.ttf]{calibri-regular.ttf}

\usepackage{bookmark}

\usepackage{hyperref}
\hypersetup{colorlinks, citecolor=black, filecolor=black, linkcolor=black, urlcolor=black}

\usepackage[english]{babel}
\usepackage{csquotes}
\usepackage{comment}

\usepackage[
    backend=biber,
    style=ieee,
    sorting=none]{biblatex}
\addbibresource{bibliography.bib}
\usepackage{tabularx}

\renewcommand{\baselinestretch}{1.15} 

\title{Project: Makerere Passion Fruit Disease classification and Detection using deep learning based Generative Adversarial Network (GAN)}
\author{Oliver Strong n11037580@qut.edu.au}
\date{\today}

\begin{document}

\begin{titlepage}
    \begin{center}
        \vspace*{1cm}
        {\huge \textbf{Makerere Passionfruit Disease Classification and Detection using Deep Learning based Generative Adversarial Networks (GANs)}}
        
        \vspace{0.5cm}
        {\Large EGH400-1 Assessment 1
        
        Project Proposal: Scope of Work}

        \vspace{0.5cm}
        \begin{tabular}{ll}
            Date & \today \\
            Student Engineer & Oliver Strong \\
            Student Number & 11037580 \\
            Supervisor & J. Banks \& K. Al-Dulaimi\\
        \end{tabular}
    \end{center}
    \begin{tabularx}{\textwidth}{@{}lllX@{}}
        \toprule
        Version & Date & Author & Changes/Comments\\
        \midrule
        1.0 & \today & Oliver Strong & \\
        \bottomrule
    \end{tabularx}
    
    \tableofcontents
\end{titlepage}

\newpage
\section{General Objective}
[One paragraph describing the overall project objective and research questions].

This project seeks to create a state-of-the-art model with a Generative 
Adversarial Network (GAN) for the detection, classification, and 
segmentation of disease on Makerere Passionfruit. The research question
that is being answered is "How can a GAN be used to detect, classify, 
and segment disease on Makerere Passionfruit and how does it compare 
to traditional methods?".


\section{Key Finding from the Literature}
[Here you should succinctly state a key takeaway / most relevant learning from your review of
literature (which will be included as an appendix)].

\section{Stakeholders \& Resources}
[Describe the end users, team members or other people who must be consulted or informed about
project progress or outcomes. Identify resources that will be used to complete the project]

The stakeholders for this project are myself, the student, the supervisors Dr. Jasmine Banks, 
and Dr. Khamael Abbas Khudhair Al-Dulaimi who will provide guidance and feedback
throughout the project. Additionally, the university is a stakeholder as the research
and all outcomes will be owned by the university. The eventual model that the project
results in will also be useful to farmers and agriculturalists in the Ugandan 
region who may look to classify potential diseases on their passionfruit crops.

Resources required for this project include access to a computer with a GPU for 
model training, access to available training data, and access to tools such as 
Python, PyTorch, and Jupyter Notebooks for model development.

This project will also require a significant amount of time to be spent on research,
and time from the project supervisors for guidance and feedback.


\section{Project Methodology}
[Describe what methodology and methods you will use to conduct your capstone research project.
When considering your approach to project methodology, you may wish to revisit your learning from
EGH404 Research in Engineering Practice. Consider what data, information, or measurements you
may collect and how you will analyse and synthesise that into a final outcome.]

\section{Deliverables}
[Identify what will be produced by the end of your project which may include constructed systems,
reports, drawings, designs, reviews, reports, presentations etc. Note that you should include interim
deliverables including status reports to your supervisor, clients, industry partners or stakeholders as
required]

\section{Risks, Requirements \& Constraints}
[Identify any existing systems that must be interfaced with. Are there any regulatory constraints?
What safety and ethical concerns will have to be considered? Are there limits to the scope of your
project because of dependencies?]
[Identify any risks including the likelihood and consequence and specify how you will mitigate or
monitor that risk]

\begin{enumerate}
\item Risk of non completion 
\item Risk of resulting model being surpassed by other work
\item Reputational risk given minority opinion on generative models
\item Risk of insufficient data to create a model that can generate sufficiently high resolution images 
\end{enumerate}

Ethical consideration will be given to the potential outcomes of the technique 
being used for malicious purposes, because the project may result in techniques 
for improved minority class detection and segmentation.

\section{Quality \& Sustainability}
[How will you measure the quality of your outcome or deliverables? How will you test the validity of
your solution? What is the scope of sustainability that you will consider, i.e. are you only interested in
the short term impact or will you have to consider life cycle analysis?]



\section{Timeline \& Deliverables}
[Outline the broad phases of your project (milestones) and the deliverables that will be achieved at
each stage (This plan should extend across both EGH400-1 and EGH400-2)]. Include designs, reports,
physical systems, etc. The following table contains some examples of what might appear in this table.



\section{Management of Project Changes}
[Outline how you will manage any stakeholder, partner, or supervisor requests for change. Note if
there are changes as the project progresses, then you should update this scope and update the
version outlining the nature of the change]

\end{document}